\begin{frontmatter}
    \title{
        Threshold Behavior  of a Vector-Host Model
    }
    \author[add:conacyt_unison]{%
    Sa\'ul D\'iaz-Infante%
    \corref{corresponding_author}
    }%
    \ead{
        saul.diazinfante@unison.mx%
    }
%%%%%%%%%%%%%%%%%%%%%%%%%%%%%%%%%%%%%%%%%%%%%%%%%%%%%%%%%%%%%%%%%%%%%%%%%%%%%%%%
    \author[add:unison]{%
        Manuel Adrian Acu\~na-Zegarra
    }%
    \ead{m.adrian.a.z@gmail.com}
%%%%%%%%%%%%%%%%%%%%%%%%%%%%%%%%%%%%%%%%%%%%%%%%%%%%%%%%%%%%%%%%%%%%%%%%%%%%%%%%
    \address[add:conacyt_unison]{
        CONACYT-Universidad de Sonora, 
        Departamento de Matem\'aticas, Boulevard Luis Encinas y 
        Rosales S/N, 83000, Hermosillo, Sonora, M\'exico.
    }
    \cortext[corresponding_author]{Corresponding author}
%
    \address[add:unison]{
        Departamento de Matem\'aticas, Universidad de Sonora, Boulevard
        Luis Encinas y Rosales S/N, Col. Centro, Hermosillo, Sonora, 
        M\'exico.
    }
    \begin{abstract}
        We present a stochastic epidemic model with vector-host structure. To 
        include environmental noise, we stochastically perturb biting rates 
        with general state functional intensities. So, we derive a stochastic 
        differential equation (SDE) which describes a vector disease with two 
        types of hosts\textemdash humans and animals\textemdash and give 
        conditions to assure disease extinction and persistence. Finally, via 
        numerical experiments, we extend and illustrate our results using 
        literature parameters for Chagas.
    \end{abstract}
    \begin{keyword}
        vector-host,
        persistence,
        extinction,  
        stochastic 
        perturbation.
    \end{keyword}
\end{frontmatter}