Our analysis needs the following function and conditions.
\begin{equation}
    \rho (S_V, I_V, I_H) := \ln (a I_V + b I_H) .
\end{equation}

\begin{theorem}[Extinction by noise]
        If 
        $
            \displaystyle
            \min
                \left \{
                      \sigma_V, \sigma_H 
                \right\} 
                > 
            \max 
                \left \{
                    \sqrt{\frac{\beta_V}{\mu_V}},
                    \sqrt{\frac{\beta_H}{\mu_H}}
                \right \}
        $,
        then the disease will extinguish with probability one.  That is,
        for any initial condition 
        $(S_V(0), I_V(0), I_H(0)) ^{\top} \in \mathbb{D}$,
        $$
            \limsup_{t \to \infty} 
                \frac{1}{t} I_V(t) = 0 
                \quad \text{and} \quad
            \limsup_{t \to \infty} 
                \frac{1}{t} I_H(t) = 0, \quad 
                \text{
                   a.s.
                }
        $$
\end{theorem}        
\begin{proof}
    The main idea is, to use a conveniently function and derive the 
    given conditions. 
    
    Let $\rho (S_V, I_V, I_H):= \ln (I_V)$, then the It\^{o} formula 
    gives
    \begin{equation} \label{eqn:ito_noise_extinction}
        \begin{aligned}
            d \ln(I_V) = & 
                \left(
                    \beta_v \frac{S_V I_H}{ I_V} - \mu_V
                    - \frac{1}{2}
                    \left(
                        \frac{S_V I_H}{I_V}
                    \right) ^ 2
                \right) dt
                %\\
                %&
                 + \sigma_V 
                \left(
                    \frac{S_V I_H}{ I_V}
                \right) dB_t ^ V.
        \end{aligned}
    \end{equation}
    %
    %
    %
    Summing an appropriate zero and applying the operator
    $$
        \left < \cdot \right >_t
        := 
        \int_{0} ^ t
            \cdot \ ds,
    $$
    to both sides of relation \eqref{eqn:ito_noise_extinction}, we get
    \begin{equation} \label{eqn:applying_mean_opeartor}
        \begin{aligned}
            \frac{1}{t} \ln(I_V(t))
            =&
            - 
            \frac{\sigma_V ^ 2}{2t}
            \int_{0} ^ {t}
                \left(
                    \frac{S_V I_H}{I_V} 
                    -
                    \left(
                        \frac{\beta_V}{\sigma_V ^ 2}
                    \right) ^ 2
                \right)
            du
            \\
            & + 
            \frac{\beta_V}{\sigma_V ^ 2}
            - 
            \mu_V 
            + 
            \underbrace{
                \frac{1}{t} \ln(I_V(0))
                +
                \frac{\sigma_V}{t}
                \int_{0}^t
                    \frac{S_V I_H}{I_V}
                    dB_u^V
            }_{:=M_t^V} .                  
        \end{aligned}
    \end{equation}
%
    Since the integral in the term $M_t^V$ is a martingale, 
    the strong law of large numbers for martingales \cite{Mao} implies that
    $$
        \lim_{t \to \infty} M_t^V = 0, \quad \text{a.s.}
    $$
    Thus, from inequality \eqref{eqn:applying_mean_opeartor} we obtain
    \begin{equation}\label{eqn:bound_vector_noise}
        \limsup_{t \to \infty} \ln(I_V(t)) 
            \leq
            \frac{\beta_V}{\sigma_V^2} - \mu_V.           
    \end{equation}
    Hence, applying condition for $\sigma_V$ and $\sigma_H$, we obtain the 
    desired conclusion.
\end{proof}
%
%
%
\paragraph{About Assumption}
The above results gives sufficient conditions to assure 
disease extinction\textemdash given sufficiently large noise. Thus, 
to propose a threshold parameter similar  to the deterministic
$\mathcal{R}_0^D$, in what follows we ask that the noise amplitudes 
$\sigma_V$, $\sigma_H$ are sufficiently small, that is,
\begin{equation}\label{eqn:noise_small_condition}
    \displaystyle
    \max
        \left \{
              \sigma_V, \sigma_H 
        \right\} 
        < 
    \min 
        \left \{
            \sqrt{\frac{\beta_V}{\mu_V}},
            \sqrt{\frac{\beta_H}{\mu_H}}
        \right \}.
\end{equation}
%
%
%
\begin{assumption}\label{ass:extinction}
    According to the vector-host model \eqref{eqn:sto_vector_host} and 
    function
    $\rho$
    we ask the following.
    \begin{enumerate}[\bf{(E-}1)]
        \item 
            The constants $a$, $b$ of function $\rho$ are positive.
        \item \label{ass:noise_condition}
            The noise intensities satisfies
            $$
                \sigma_V \leq 
                    \sqrt{
                        \frac{b}{a}
                        \frac{\beta_V}{N_V^{\infty}}
                        },
                 \quad
                \sigma_H \leq
                    \sqrt{
                        \frac{a}{b}
                        \frac{\beta_H }{N_H
                    }}.
            $$
        \item
            \begin{equation*}
                \begin{aligned}
                    \mathcal{R}_0 ^ S &:=
                        \mathcal{R}_0^D - 
                        \left( 
                            \frac{\sigma_V ^ 2}{2} (N_V ^ {\infty}) ^ 2
                            +
                            \frac{\sigma_H ^ 2}{2} N_H ^ 2
                        \right)
                        <1,
                        \\
                    \mathcal{R}_0 ^ D &:=
                            \frac{
                                \beta_V \beta_H N_V ^ \infty N_H}{\mu_V \mu_H} .
                \end{aligned}
            \end{equation*}
    \end{enumerate}
\end{assumption}        
%
%
%
%
\begin{proposition}\label{prp:quadratic_bound}
    According to SDE model \eqref{eqn:sto_vector_host},
    let $\varphi = \varphi (S_V, I_V, I_H)$, 
    $\psi = \psi (S_V, I_V, I_H)$ 
    functions defined by
    \begin{equation}
        \begin{aligned}
            \varphi (S_V, I_V, I_H) &:= 
                \beta_V \frac{S_V I_H}{a I_V + b I_H}
                - 
                \frac{\sigma_V ^ 2}{2} 
                \left(
                    \frac{S_V I_H}{a I_V + b I_H}
                \right) ^2,
            \\
            \psi (S_V, I_V, I_H) &:=
            \beta_H \frac{(N_H - I_H)  I_V}{a I_V + b I_H}
                - 
                \frac{\sigma_H ^ 2}{2} 
                \left(
                    \frac{(N_H - I_H)  I_V}{a I_V + b I_H}
                \right) ^2.
        \end{aligned}
    \end{equation}
    If the noise intensities $\sigma_V$, $\sigma_H$ satisfies condition 
    (E\textendash\ref{ass:noise_condition})
    of \Cref{ass:extinction}, then
    \begin{equation}
        \begin{aligned}
          \varphi (S_V, I_V, I_H) & \leq \beta_V \frac{a}{b} N_V ^ 
          {\infty}
         - \frac{\sigma_V ^ 2}{2}
          \left(
                  \frac{a}{b} N_V ^ {\infty}
              \right) ^ 2,
          \\
          \psi (S_V, I_V, I_H) & \leq
              \beta_H \frac{b}{a} N_H - \frac{\sigma_H ^ 2}{2}
              \left(
                  \frac{b}{a} N_H
              \right) ^ 2,
        \end{aligned}
    \end{equation}
    for all $(S_V, I_V, I_H)^{\top}$ in the invariant set $\mathbb{D}$.
\end{proposition}
%
\paragraph{Intro to the extinction Theorem}
\begin{lemma}
    Let \Cref{ass:extinction} hold, 
    and assume that 
    then there is a positive integer number $n$
    such that the equation
    \begin{equation}\label{eqn:quadratic_extinction}
        2^n
        \left(
            \beta_V N_V^{\infty} +
            \beta_H N_H
        \right) x
        - \frac{\mu_V \mu_H}{x}
        =
        \left(
            \mathcal{R}_0^D -1
        \right),
    \end{equation}
    has a positive solution $x=x^\dagger$. 
\end{lemma}
\begin{proof}
    We can rewrite \eqref{eqn:quadratic_extinction} as
    \begin{equation*}
        \left(
            \beta_V N_V ^ {\infty} +
            \beta_H N_H            
            -\frac{\mathcal{R}_0 ^ D - 1}{2 ^ n}
        \right) x ^ 2
        - \frac{\mu_v \mu_H}{2 ^ n}
        = 0.
    \end{equation*}
    Thus, the solution of \eqref{eqn:quadratic_extinction} is
    $$
        x^{\dagger} = 
            \frac{\mu_v \mu_H}{
                \left(
                    \beta_V N_V ^ {\infty} +
                    \beta_H N_H            
                    -\frac{\mathcal{R}_0 ^ D - 1}{2 ^ n}
                \right)
            }.
    $$
    To assure the positivity of $x^{\dagger}$ we consider two cases.
    If $\mathcal{R}_0 ^ D \leq 1$, 
    then $\mathcal{R}_0 - 1 \leq 0$, and $x^{\dagger} >0$.
    Otherwise, if $\mathcal{R}_0 ^ D \geq 1$,
    then taking 
    $$          
        n > \frac{1}{\ln 2} 
            \ln
            \left(
                \displaystyle
                \frac{\mathcal{R}_0 ^ D}{
                    \left(
                        \beta_V N_V ^ {\infty} +
                        \beta_H N_H
                    \right)
                }    
            \right),
    $$
    we obtain the positivity of $x^{\dagger}$.
\end{proof}
%
%
%
Now we establish our threshold extinction result.
\begin{theorem}
    If \Cref{ass:extinction} hold, then the solution of model 
    \eqref{eqn:sto_vector_host} satisfies
    \begin{equation}
        \limsup_{t \to \infty}
            \ln(aI_V + b I_H) < c , \qquad a.s.,
    \end{equation}
    that is, the disease will extinguishes wit probability one.
\end{theorem}
%
%
\begin{proof}
    The main idea is to apply \Cref{prp:quadratic_bound} and 
    \Cref{ass:extinction} to the It\^{o} formula of $\rho(S_V, I_V, I_H)$.
\end{proof}
