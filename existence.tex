\begin{theorem}\label{thm:regularity}
	Under \Cref{ass:regularity}, there is a unique solution $\left( 
	S_{V}(t),I_{V}(t),I_{H}(t) \right)$ to SDE \eqref{eqn:sto_vector_host} 
	for $t\geq 0$ and this solution remains in $\mathbf{D}$ a.s., that is, 
	$\left( S_{V}(t),I_{V}(t),I_{H}(t) \right)\in \mathbf{D}$ for all $t\geq 0$ 
	with probability one.
\end{theorem}
%
%
%
\begin{proof}
    Let 
    $
        X(t) = \left( S_{V}(t), I_{V}(t), I_{H}(t) \right)
    $.
    For each $n\in \mathbb{N}$ define,
    \begin{align*}
        \mathbf{D}_{n}
        := &
    \left
    \lbrace
	X
	\in \mathbb{R}^{3}:
	e^{-n} < S_{V} < N_{V}^{\infty}-e^{-n},
	\enskip
	e^{-n} < I_{V} < N_{V}^{\infty}-e^{-n},
	\enskip
	S_{V} + I_{V} \leq N_{V}^{\infty},
	\right.\\
	&\ 
	\left.
%	\enskip
	e^{-n} < I_{h} < N_{H}-e^{-n},
	\enskip
	S_{V} + \alpha I_{H} \leq N_{V}^{\infty}
        \right\rbrace.
	\end{align*}
	Since the drift and diffusion of SDE \eqref{eqn:sto_vector_host} are 
	locally Lipschitz-continuous and satisfy linear growth condition on 
	$\mathbf{D}_{n}$, so there is a unique local solution on 
	$[0,\tau(\mathbf{D}_{n}))$ for any initial condition $X\left(0\right)\in 
	\mathbf{D}_{n}$. Here, $\tau(\mathbf{D}_{n})$ represents the random time of 
	the first exit of the stochastic process $X\left(t\right)$ from the set 
	$\mathbf{D}_{n}$.
	
	To prove global existence of the solution, we follow ideas of 
	\cite{Schurz2015, Acuna2018}. 
	Let,
	\begin{align*}
		V(X) 
		&:= \left(N_{V}^{\infty} - S_{V}\right) - \ln\left(N_{V}^{\infty} - 
		S_{V}\right) + S_{V} - \ln S_{V} + I_{v} + \left(N_{V}^{\infty} - 
		I_{V}\right) - \ln\left(N_{V}^{\infty} - I_{V}\right) + I_{H} +
		\left(N_{H} - I_{H}\right) - \ln\left(N_{H} - I_{H}\right),
	\end{align*}
	defined on
	$
		\widetilde{\mathbf{D}} 
		:=\{ 
		X
		\left( 
		t
		%I_{h}, I_{a}, S_{v},I_{v} 
		\right) \in \mathbb{R}^{3};  
		\ 0\leq t: \ 
		0< S_{V}< N_{V}^{\infty},
		0< I_{V}< N_{V}^{\infty},
		\ 
		S_{V}+I_{V}\leq N_{V}^{\infty},
		0 < I_{H}< N_{H},
		\ 
		S_{V} + \alpha I_{H}\leq N_{V}^{\infty}
		\}.
	$
	Proceeding in a similar manner to the existence demonstration in 
	\cite{Acuna2018}, it follows that $V (X) \geq 4$ for all $X	\in 
	\widetilde{\mathbf{D}}$.
	
	Applying the infinitesimal generator to the Lyapunov function 
	$V\left( X\right)$, we obtain
	\begin{align*}
		\mathcal{L}V(X) = &
			\left[\Lambda_{V} - \beta_{V} S_{V} I_{H} - \mu_{V} S_{V} 
			\right]\frac{\partial V}{\partial S_{V}}
			+
			\left[\beta_{V} S_{V} I_{H} - \mu_{V} 
			I_{V}\right]\frac{\partial V}{\partial I_{V}}
			+
			\left[\beta_{H} \left(N_{H} - I_{H}\right) I_{V} - \mu_{H} 
			I_{H}\right]\frac{\partial V}{\partial I_{H}}\\
			&\ 
			+
			\frac{1}{2}\sigma_{V}^{2}S_{V}^{2}I_{H}^{2}\left(\frac{\partial^{2}V}{\partial
			 S_{V}^{2}} + \frac{\partial^{2}V}{\partial I_{V}^{2}}\right)
			+
			\frac{1}{2}\sigma_{H}^{2}\left(N_{H} - 
			I_{H}\right)^{2}I_{V}^{2}\frac{\partial^{2}V}{\partial
			I_{H}^{2}}\\[0.2cm]
			= & 
			\frac{\Lambda_{V} - \beta_{V} S_{V} I_{H} - \mu_{V} 
			S_{V}}{N_{V}^{\infty} - S_{V}}
			+
			\frac{\Lambda_{V} - \beta_{V} S_{V} I_{H} - \mu_{V} S_{V}}{- 
			S_{V}}
			+
			\frac{\beta_{V} S_{V} I_{H} - \mu_{V}I_{V}}{N_{V}^{\infty} - 
			I_{V}}
			+
			\frac{\beta_{H} \left(N_{H} - I_{H}\right) I_{V} - \mu_{H} 
			I_{H}}{N_{H} - I_{H}}\\
			&\ 
			+
			\frac{1}{2}\sigma_{V}^{2}S_{V}^{2}I_{H}^{2}\left(\frac{1}
			{\left(N_{V}^{\infty} - S_{V}\right)^{2}} + \frac{1}
			{S_{V}^{2}} + \frac{1}{\left(N_{V}^{\infty} - 
			I_{V}\right)^{2}}\right)
			+
			\frac{1}{2}\sigma_{H}^{2}\left(N_{H} - 
			I_{H}\right)^{2}I_{V}^{2}\left(\frac{1}{\left(N_{H} - 
			I_{H}\right)^{2}}\right)~.
	\end{align*}
	Then, removing some negative terms of the above expression, it follows
	\begin{align*}
	\mathcal{L}V(X) \leq &
		\frac{\Lambda_{V} - \mu_{V}S_{V}}{N_{V}^{\infty} - S_{V}}
		+
		\beta_{V}I_{H} + \mu_{V}
		+
		\frac{\beta_{V} S_{V} I_{H}}{N_{V}^{\infty} - I_{V}}
		+
		\beta_{H}I_{V}
		+
		\frac{1}{2}\left(\frac{\sigma_{V}^{2}S_{V}^{2}I_{H}^{2}}
		{\left(N_{V}^{\infty} - S_{V}\right)^{2}} + \sigma_{V}^{2}I_{H}^{2} + 
		\frac{\sigma_{V}^{2}S_{V}^{2}I_{H}^{2}}{\left(N_{V}^{\infty} - 		
		I_{V}\right)^{2}} + \sigma_{H}^{2}I_{V}^{2}\right)~.
	\end{align*}
	Since 
    $
        \displaystyle
        N_{V}^{\infty} = 
            \frac{
                \Lambda_{V}
            }{\mu_{V}},
        \quad 
        S_{V} + I_{V}
        \leq 
            N_{V}^{\infty},
        \quad 
        S_{V} + \alpha I_{H}
        \leq 
            N_{V}^{\infty},
    $ 
    wee see that
    \begin{align*}
        \mathcal{L}V(X) 
        \leq &
            \ \mu_{V}
            +
            \beta_{V}I_{H} + \mu_{V}
            +
            \beta_{V} I_{H}
            +
            \beta_{H} I_{V}
            +
            \frac{1}{2}
            \left(
                \frac{
                    \sigma_{V}^{2} S_{V}^{2}
                }
                {
                    \alpha^{2}} 
                    + \sigma_{V} ^ {2}
                    I_{H} ^ {2} 
                    + \sigma_{V} ^ {2} I_{H}^{2} 
                    + \sigma_{H} ^ {2} I_{V}^{2}
            \right)
            \\
            \leq 
            &
            \underbrace{
                2\mu_{V}
                +
                2 \beta_{V} N_{H} 
                +
                \beta_{H}
                N_{V} ^ {\infty}
                +
                \frac{1}{2}
                \left[
                    \left(
                        \frac{ 
                            \sigma_{V}
                            N_{V} ^ {\infty}
                        }{
                            \alpha}
                    \right) ^ {2} 
                    +2
                    \left(
                        \sigma_{V} 
                        N_{H}
                    \right) ^ {2} 
                    + 
                    \left(
                        \sigma_{H}
                        N_{V} ^ {\infty}
                    \right) ^ {2}
                \right]
            }_{:= 4c} ~.
    \end{align*}
    Since $V(X) \geq 4$ for all $X \in \widetilde{\mathbf{D}}$, 
    using the above inequality, we deduce that 
    $
        cV 
        \left(
            X 
        \right)
        \geq 
        \mathcal{L} V
        \left(
             X 
        \right)
    $.
    
    Now, for each $n \in \N$, we construct a crescent collection of subsets of 
    $\mathbf{D}$,
    $$
        \mathbf{D}_{n} :=
        [e^{-n}, N_{V}^{\infty} - e^{-n}] \times
        [e^{-n}, N_{V}^{\infty} - e^{-n}] \times
        [e^{-n}, N_{H} - e^{-n}].
    $$
    Then
    $$
        \widetilde{\mathbf{D}}
        \backslash\mathbf{D}_{n} 
        = 
        \left(
            0,e^{-n}
        \right)
        \bigcup
        \left(
            N_{V}^{\infty} - e^{-n},N_{V}^{\infty}
        \right)
	\times 
	\left(0,e^{-n}\right)
	\bigcup\left(N_{V}^{\infty} - e^{-n},N_{V}^{\infty}\right)
	\times
	\left(0,e^{-n}\right)
	\bigcup\left(N_{H} - e^{-n},N_{H}\right) .
	$$
	On the other hand, as $f_{aux}(x) = x - \ln x > 1 \ \forall x > 0$, so 
	$V\left(X\right)\geq (N_{V}^{\infty} - S_{v}) - \ln(N_{V}^{\infty} - S_{v}) 
	+ S_{v} - \ln S_{v} + 2, \text{ for all }X\in \widetilde{\mathbf{D}} 
	\backslash \mathbf{D}_{n}$. Furthermore, since $f_{aux}(\cdot)$ is 
	an increasing function in $(K-e^{-n},K)$, if follows that
	\begin{equation*}
	\begin{aligned}
	V(X)
	&\geq 
	(N_{V}^{\infty} - N_{V}^{\infty} + e^{-n})- \ln(N_{V}^{\infty} - 
	N_{V}^{\infty} + e^{-n}) + 3\\
	&\geq e^{-n} + n + 3\\
	&> n + 3.
	\end{aligned}
	\end{equation*}
	Similarly to the above, as $f_{aux}(\cdot)$ decreases in $(0,e^{-n})$, so
	\begin{equation*}
	\begin{aligned}
	V(X)
	& \geq e^{-n}- \ln(e^{-n}) + 3 \\
	& \geq e^{-n} + n + 3 \\
	& > n + 3.
	\end{aligned}
	\end{equation*}
	Hence, we infer that $V\left(X\right) > n + 3$ for all $X	\in 
	\widetilde{\mathbf{D}} \backslash\mathbf{D}_{n}$, and consequently
	\begin{equation}\label{eqn:v_condition}
	\inf_{
		X
		\in 
		\widetilde{\mathbf{D}}
		\backslash
		\mathbf{D}_{n}}
	V
	\left( X \right) 
	> n + 3, \quad \text{for all } n\in \N ~.
	\end{equation}
	
	Next, we prove that $X(t)$ remains in $\widetilde{\mathbf{D}}$. Let $W 
	\left(t,X(t)\right) = e^{-c(t-t_{0})}V\left(X\right)$ defined on 
	$[t_{0},\infty) \times \widetilde{\mathbf{D}}$ and $t_{0}\geq 0$ fixed. 
	Thus, since $\mathcal{L}V\left( X	\right)	\leq cV\left( X	\right)$, we 
	have that $\mathcal{L}W\left( t,X \right)\leq 0$ for $(t, X(t))\in 
	[t_{0},\infty) \times \widetilde{\mathbf{D}}$. Now, to obtain an upper 
	bound for $\mathbb{E}W\left( \tau_{n},X(\tau_{n})\right)$, where 
	$\tau_{n}(t) := \min \{ t,\tau(\mathbf{D}_{n})\}$, we apply the Dynkin 
	formula, then
	\begin{equation} \label{eqn:Dynkin_bound}
		\begin{aligned}
			\mathbb{E}
			W
			\left( 
			\tau_{n},
			X(\tau_{n})
			\right) 
			=& \mathbb{E}
			W
			\left( 
			t_{0}, X(t_0)
			\right)
			+ 
			\mathbb{E}
			\int_{t_{0}}^{\tau_{n}}
			\mathcal{L} W 
			\left(
			s, X(s)
			\right)ds
			\\ %
			\leq& 
			\mathbb{E}
			W
			\left( 
			t_0, X(t_0)
			\right)
			\\ %
			=& 
			\mathbb{E}
			V
			\left( 
			X(t_0)
			\right).
		\end{aligned}
    \end{equation}
    Then, using the inequality~\eqref{eqn:Dynkin_bound} and that $\mathbf{D}_{n}
    \subset \mathbf{D}_{n+1}$, it follows
    \begin{align*}
        \mathbb{P}
        \left(
            \tau(\widetilde{\mathbf{D}}) < t 
        \right)
        &\leq 
        \mathbb{P}
        \left(
            \tau(\mathbf{D}_{n})<t 
        \right)
        \\ %
        & = 
        \mathbb{E}
        \left(
            \1{\tau_{n} < t}
        \right)
        \\ %
        &\leq 
        \mathbb{E}
        \left( 
            e ^ {c(t - \tau_{n})}
            \frac{
                V
                \left(
                    X(\tau_{n})
                \right)
            }{
                \inf
                    \limits_{
                        X \in 
                        \widetilde{\mathbf{D}}
                        \backslash \mathbf{D}_{n}
                    }
                V
                \left( 
                    X
                \right)} 
                \1{\tau_{n}<t}
        \right)
        \\
        &\leq 
        \frac{
            e^{ c (t - t_{0} )}
            \mathbb{E}V
            \left( 
                X(t_0)
            \right)
        }
        {
            \inf
            \limits_{
                X
                \in
                \widetilde{\mathbf{D}}
                \backslash 
                \mathbf{D}_{n}} 
                V
            \left( 
                X
            \right)
        }
        \\ %
        &\leq
        \frac{
            e^{c(t-t_{0})}
            \mathbb{E} V
            \left(
                X(t_0)
            \right)
        }{
            n + 3
        },
\end{align*}
then, letting $n \to \infty$, gives $\mathbb{P}(\tau_{n}<t)\rightarrow 0$, 
for all $\left( X(t_{0})\right)\in \mathbf{D}_{n}$ and fixed $t\in 
[t_{0},\infty)$. Furthermore, by the above inequality, we conclude that 
$\mathbb{P}(\tau(\widetilde{\mathbf{D}})=\infty)=1$.

Thus, we have proved that, for all $t \geq 0$ and initial condition on 
$\widetilde{\mathbf{D}}$, the invariance property and the global existence of 
the solution $X(t)$ on $\widetilde{\mathbf{D}}$ is satisfied with probability 
one.\todo{Falta agregar la continuidad pero agregando los teoremas a usar en el 
apéndice}

Note that the cases $I_{H} = 0$, $S_{V} = N_{V}^{\infty}$ and $I_{V} = 0$ are 
not in $\widetilde{\mathbf{D}}$. So, we discuss these cases below.
\begin{enumerate}[{(CASE-}I)]
	\item 
	We first study the case $I_{H} = 0$. With this assumption, 
	SDE~\eqref{eqn:sto_vector_host} becomes an ODE
	\begin{equation}\label{IH:0}
		\begin{aligned}
			dS_{V} 
			=&
			\left[
			\Lambda_{V} - \mu_{V}S_{V}
			\right]dt 
			\\
			dI_{V} 
			=&
			- \mu_{V} I_{V}dt~.
		\end{aligned}
	\end{equation}
	Note that \Cref{IH:0} is lineal, therefore has a unique continuous global 
	solution in 
	$$
		\mathbf{D}_{1} := 
		\left\{ 
		\left( 
		S_{V}(t),
		I_{V}(t) 
		\right)
		\in \mathbb{R}^{2};
		\ 0\leq t: \ 0< S_{V} < N_{V}^{\infty},\ 
		0< I_{V} < N_{V}^{\infty},\ 
		S_{V} + I_{V}\leq N_{V}^{\infty}
		\right\}.
		$$
%
	\item
	When $I_{V} = 0$, the SDE~\eqref{eqn:sto_vector_host} becomes
	\begin{equation} \label{IV:0}
		\begin{aligned}
			d S_{V} &= 
			\left [
			\Lambda_{V} - \beta_{V} S_{V} I_{H} - \mu_{V} S_{V} 
			\right ] dt
			- \sigma_{V} S_{V} I_{H} dB_{t}^{V},
			\\
			d I_{H} &= 
			- \mu_{H} I_{H} dt .
		\end{aligned}
	\end{equation}
		Let 
		$
		\mathbf{D}_{2} := 
		\left\{ 
		\left( 
		S_{V}(t),
		I_{H}(t) 
		\right)
		\in \mathbb{R}^{2};
		\ 0\leq t :
		\ 0<S_{V}< N_{V}^{\infty},
		\ 0< I_{H}< N_{H},		 
		\ S_{V} + \alpha I_{H} \leq N_{V}^{\infty}
		\right\}
		$ 
		the set over which is defined the SDE \eqref{IV:0}, and the Lyapunov 
		function
		$$
		V_{1}
		\left( 
		S_{V},I_{H} 
		\right) 
		= \left(N_{V}^{\infty} - S_{V}\right) - \ln\left(N_{V}^{\infty} - 
		S_{V}\right) + S_{V} - \ln S_{V} + I_{H} + \left(N_{H} - I_{H}\right) - 
		\ln\left(N_{H} - I_{H}\right).
		$$
		Following the same ideas of the main demonstration, we can prove the 
		invariance of $\mathbf{D}_{2}$, global existence, continuity and 
		uniqueness of the solution.
%
	\item
	Now, if $S_{V} = N_{V}^{\infty}$, since $S_{V} + I_{V} \leq N_{V}^{\infty}$ 
	and $S_{V} + \alpha I_{H} \leq N_{V}^{\infty}$, then $I_{V} = 0$ and $I_{H} 
	= 0$. This last implies that the solution of 
	SDE~\eqref{eqn:sto_vector_host} is the disease-free equilibrium.
\end{enumerate}

Therefore, the proof is complete.
\end{proof}