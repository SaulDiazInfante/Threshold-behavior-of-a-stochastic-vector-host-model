Our analysis needs the following function and conditions.
\begin{equation}
    \rho (S_V, I_V, S_H, I_H)            := \ln (a I_V + b I_H) .
\end{equation}

\begin{theorem}[Extinction by noise]
    If 
    $
        \displaystyle
        \min
            \left \{
                  \sigma_V, \sigma_H 
            \right\} 
            > 
        \max 
            \left \{
                \sqrt{
                    \frac{\beta_V ^ 2}{2 \mu_V}
                },
                \sqrt{
                    \frac{\beta_H ^ 2}{2 \mu_H}
                }
            \right \}
    $,
        then the disease will extinguish with probability one.  That is,
        for any initial condition 
        $(S_V(0), I_V(0), I_H(0)) ^{\top} \in \mathbb{D}$,
        $$
            \limsup_{t \to \infty} 
                \frac{1}{t} \ln I_V(t) = 0 
                \quad \text{and} \quad
            \limsup_{t \to \infty} 
                \frac{1}{t} \ln I_H(t) = 0, \quad 
                \text{
                   a.s.
                }
        $$
\end{theorem}        
\begin{proof}
    The main idea is apply the It\^{o} formula to a conveniently function and 
    deduce conditions. 
    
    Let $\rho (S_V, I_V, S_H, I_H) := \ln I_V$, then the It\^{o} 
    formula gives
    \begin{equation} \label{eqn:ito_noise_extinction}
        \begin{aligned}
            d \ln I_V = & 
                \left(
                    \beta_V
                    \frac{S_V I_H}{ I_V} - \mu_V
                    - 
                    \frac{1}{2}
                    \left(
                        \frac{\sigma_V}{N_V^{\infty}}
                    \right) ^ 2
                    \left(
                        \frac{ S_V I_H}{I_V}
                    \right) ^ 2
                \right) dt
                %\\
                %&
                 + 
                \frac{\sigma_V}{N_V^{\infty}}
                \left(
                    \frac{S_V I_H}{ I_V}
                \right) dB_t ^ V.
        \end{aligned}
    \end{equation}
    %
    %
    %
    Summing an appropriate zero and applying the mean operator
    $$
        \left < \cdot \right >_t
        := 
            \frac{1}{t}
            \int_{0} ^ t
                \cdot \ du,
    $$
    to both sides of relation \eqref{eqn:ito_noise_extinction}, we get
    \begin{equation} \label{eqn:applying_mean_opeartor}
        \begin{aligned}
            \frac{1}{t} \ln(I_V(t))
            =&
            -
            \frac{1}{2t}
            \left(
                \frac{\sigma_V }{N_V^{\infty}}
            \right) ^ 2
            \int_{0} ^ {t}
                \left(
                    \frac{S_V I_H}{I_V} 
                    -
                    \frac{\beta_V}{\sigma_V ^ 2}
                \right)^2
            du
            \\
            & 
            + 
            \frac{1}{2}
            \left(
                \frac{\beta_V}{\sigma_V}
            \right) ^ 2
            -
            \mu_V 
            + 
            \underbrace{
                \frac{1}{t} \ln(I_V(0))
                +
                \frac{\sigma_V}{t}
                \int_{0}^t
                    \frac{S_V I_H}{I_V}
                    dB_u^V
            }_{:=M_t ^ V}
            \\
            \leq
                &
            \frac{1}{2}
            \left(
                \frac{\beta_V}{\sigma_V}
            \right) ^ 2
            -
            \mu_V 
            + 
            M_t ^ V.
        \end{aligned}
    \end{equation}
    Since the integral in the term $M_t$ is a martingale, 
    the strong law of large numbers for martingales 
    {\citet[p. 12, Thm 3.4]{Mao2007}} implies that
    $$
        \lim_{t \to \infty} M_t ^ v = 0, \quad \text{a.s.}
    $$
    Thus, from relation \eqref{eqn:applying_mean_opeartor} 
    we obtain
    \begin{equation} \label{eqn:bound_vector_noise_V}
        \limsup_{t \to \infty} 
            \frac{1}{t}
            \ln(I_V(t)) 
            \leq
            \frac{1}{2}
            \left(
                \frac{\beta_V}{\sigma_V}
            \right) ^ 2
             - \mu_V.
    \end{equation}
    A similar argument also shows that
        \begin{equation}\label{eqn:bound_vector_noise_H}
            \limsup_{t \to \infty} 
                \frac{1}{t}
                \ln(I_H(t)) 
                \leq
                \frac{1}{2}
                \left(
                    \frac{\beta_H}{\sigma_H}
                \right) ^ 2
                - \mu_H.
        \end{equation}
    Hence, under condition
        $
            \displaystyle
            \min
                \left \{
                      \sigma_V, \sigma_H 
                \right\} 
                > 
            \max 
            \left \{
                \sqrt{\frac{(\beta_V N_V ^ {\infty}) ^ 2}{2 \mu_V}},
                \sqrt{\frac{(\beta_H N_H) ^ 2}{2 \mu_H}}
            \right \}
        $, 
    we obtain the desired conclusion.
\end{proof}
%
%
%
\paragraph{About Assumption}
The above results gives sufficient conditions to assure 
disease extinction\textemdash given sufficiently large noise. Thus, 
to propose a threshold parameter similar  to the deterministic
$\mathcal{R}_0^D$, in what follows we ask that the noise amplitudes 
$\sigma_V$, $\sigma_H$ are sufficiently small, that is,
\begin{equation}\label{eqn:noise_small_condition}
    \displaystyle
    \max
        \left \{
              \sigma_V, \sigma_H 
        \right\} 
        < 
    \min 
        \left \{
            \sqrt{\frac{(\beta_V N_V ^ {\infty}) ^ 2}{2 \mu_V}},
            \sqrt{\frac{(\beta_H N_H) ^ 2}{2 \mu_H}}
        \right \}.
\end{equation}
%
%
%
\begin{assumption}\label{ass:extinction}
    According to the vector-host model \eqref{eqn:sto_vector_host} and 
    function
    $\rho$
    we ask the following.
    \begin{enumerate}[\bf{(E\textendash}1)]
        \item 
            Constants $a$, $b$ of function $\rho$ are positive and
            $$
                \max \{\mu_V, \mu_H \} < \min \{a, b \}.
            $$
        \item \label{ass:noise_condition}
            Noise intensities $\sigma_V, \sigma_H$ satisfy
            $$
                \sigma_V \leq 
                    \sqrt{
                        \frac{
                            \min\{a, b\}
                        }{
                            \max \{a, b \}
                        }
                    \beta_V  N_V^{\infty}
                    },
                 \quad
                \sigma_H \leq
                    \sqrt{
                        \frac{
                            \min\{a, b\}
                        }{
                            \max \{a, b \}
                        } 
                        \beta_H N_H 
                    } .
            $$
        \item
            According to the deterministic reproductive
            number 
            $   \displaystyle
                \mathcal{R}_0 ^ D :=
                    \frac{
                         \beta_V \beta_H 
                         N_V ^ \infty N_H
                    }{
                        \mu_V \mu_H
                    }
            $,

            $$
            \mathcal{R}_0 ^ S :=
                \mathcal{R}_0^D - 
                \frac{1}{2}
                \left( 
                    \sigma_V ^ 2
                    +
                    \sigma_H ^ 2
                \right)
                <1 .
            $$
    \end{enumerate}
\end{assumption}
%
%
%
%
\begin{proposition}\label{prp:quadratic_bound}
    With the notation of SDE model \eqref{eqn:sto_vector_host},
    let
    \begin{equation}
        \begin{aligned}
            \varphi (S_V, I_V, S_H, I_H) &:= 
                \beta_V
                 \frac{a S_V I_H}{a I_V + b I_H}
                - 
                \frac{1}{2}
                \left(
                    \frac{\sigma_V }{N_V^\infty} 
                \right) ^ 2
                %
                \left(
                    \frac{ a S_V I_H}{a I_V + b I_H}
                \right) ^2,
            \\
            \psi (S_V, I_V, S_H, I_H) &:=
                \beta_H 
                \frac{
                    b S_H  I_V
                }{%
                    a I_V + b I_H
                }
                - 
                \frac{1}{2}
                \left(
                    \frac{\sigma_H}{ N_H } 
                \right) ^ 2
                    \left(
                        \frac{
                            b S_H  I_V
                        }{
                            a I_V + b I_H
                        }
                    \right) ^ 2.
        \end{aligned}
    \end{equation}
    If the noise intensities $\sigma_V$, $\sigma_H$ satisfies condition 
    \textsc{
        \textbf{(E\textendash\ref{ass:noise_condition})}
    }
    of \Cref{ass:extinction}, then
    \begin{equation}
        \label{eqn:max_min_bounds}
        \begin{aligned}
            \varphi (S_V, I_V, S_H, I_H)            
            & \leq 
                \frac{
                    \max\{a,b\}
                }{
                    \min \{a, b\}
                }
                    \beta_{V}
                  N_V ^ {\infty}
                - 
                \frac{\sigma_V ^ 2}{2}
                \left(
                    \frac{
                        \max\{a,b\}
                    }{
                        \min \{a, b\}
                    }
                \right) ^ 2,
            \\
            \psi (S_V, I_V, S_H, I_H)            
            & \leq
                \frac{
                    \max\{a,b\}
                }{
                    \min \{a, b\}
                }
                \beta_H  N_H - \frac{\sigma_H ^ 2}{2}
                    \left(
                        \frac{
                            \max\{a,b\}
                        }{
                            \min \{a, b\}
                        }
                        % N_H. N_H/N_V
                  \right) ^ 2,
        \end{aligned}
    \end{equation}
    for all $(S_V, I_V, S_H, I_H)           ^{\top}$ in the invariant set 
    $\mathbb{D}$.
\end{proposition}
\begin{proof}
    The main idea is to prove that
    \begin{equation}
        \label{eqn:phi_difference}
        \frac{
            \max\{a,b\}
        }{
            \min \{a, b\}
        }
          N_V ^ {\infty}
        - 
        \frac{\sigma_V ^ 2}{2}
        \left(
            \frac{
                \max\{a,b\}
            }{
                \min \{a, b\}
            }
        \right) ^ 2
        -
        \varphi (S_V, I_V, S_H, I_H)            
        \geq 
        0.
    \end{equation}
    We rewrite the left hand side of \eqref{eqn:phi_difference} as
%
    \begin{equation}
    \label{eqn:phi_difference_rewritten}
        \left(
            \frac{
                \max \{a,b\}
            }{
                \min \{a,b\}
            }
            N_V ^ \infty
%            
            -
            \frac{ a S_V I_H}{a I_V + b I_H}
        \right)
        \left[
            \beta_V N_V ^ {\infty}
            -
            \frac{1}{2}
            \left(
                \frac{\sigma_V}{N_V ^ {\infty}}
            \right) ^2
            \left(
                \frac{
                    \max \{a,b\}
                }{
                    \min \{a,b\}
                }
                N_V ^ {\infty}
                +
                \frac{a S_V I_H}{a I_V + b I_H}
            \right)
        \right].
    \end{equation}
%
    From \Cref{lem:exponential_growth} we see that
    $S_V \approx N_V ^ \infty$ if and only if $I_V \approx 0$. Consequently,
    \begin{equation}\label{eqn:phi_first_term_bound}
        \beta_V
            \frac{a S_V I_H}{a I_V + b I_H}
        \leq
         \beta_V
            \frac{a N_V ^ \infty I_H}{b I_H}
        \leq
            \frac{
                \max \{ a,b \} 
            }{
                \min\{ a, b\}
            }
             N_V ^ \infty.
    \end{equation}
    Thus, inequality \eqref{eqn:phi_first_term_bound} proves
    that first factor in \eqref{eqn:phi_difference_rewritten} is nonnegative.
    Further, 
    $$
        \beta_V
        -
        \frac{1}{2}
        \left(
            \frac{\sigma_V}{N_V ^ {\infty}}
        \right) ^2
        \left(
            \frac{
                \max \{a,b\}
            }{
                \min \{a,b\}
            }
            N_V ^ \infty
            +
            \frac{a S_V I_H}{a I_V + b I_H}
        \right)
        \geq
        \beta_V
        -
        \left(
            \frac{\sigma_V}{N_V ^ {\infty}}
        \right) ^2
            \frac{
                \max \{a,b\}
            }{
                \min \{a,b\}
            }
            N_V ^ \infty,
    $$
    then, condition \textsc{ \textbf{(E\textendash 1)}}, implies that
    second term in \eqref{eqn:phi_difference_rewritten} also results 
    nonnegative.
    
        The same reasoning applies to obtain the second inequality in 
    \eqref{eqn:max_min_bounds}.
\end{proof}


%
\paragraph{Intro to the extinction Theorem}
\begin{lemma}
    \label{lem:quadratic_equation}
    Let \Cref{ass:extinction} hold, 
    and assume that exist a positive integer number $n$
    such that 
    \begin{equation}\label{eqn:quadratic_extinction}
        2^n
        \left(
            \beta_V
             N_V^{\infty} +
            \beta_H N_H
        \right) 
        - \frac{\mu_V \mu_H}{x}
        =
        \mathcal{R}_0^D -1
        ,
    \end{equation}
    has a positive solution $x=x^\dagger$. 
\end{lemma}
\todo{Write the fucking argument}
%
\begin{proof}
    We can rewrite \eqref{eqn:quadratic_extinction} as
    \begin{equation*}
        \left(
             \beta_V
             N_V ^ {\infty} +
            \beta_H N_H
            -\frac{\mathcal{R}_0 ^ D - 1}{2 ^ n}
        \right) x ^ 2
        - \frac{\mu_v \mu_H}{2 ^ n} x
        = 0.
    \end{equation*}
    Thus, the solution of \eqref{eqn:quadratic_extinction} is
    \begin{equation}
        \label{eqn:quadratic_solution}
        x^{\dagger} = 
            \frac{\mu_v \mu_H}{
                \left(
                     \beta_V
                     N_V ^ {\infty} +
                    \beta_H N_H
                    -\dfrac{\mathcal{R}_0 ^ D - 1}{2 ^ n}
                \right)
            }.
    \end{equation}
    To assure the positivity of $x^{\dagger}$ we consider two cases.
    If $\mathcal{R}_0 ^ D \leq 1$, 
    then $\mathcal{R}_0 - 1 \leq 0$, and $x^{\dagger} >0$.
    Otherwise, if $\mathcal{R}_0 ^ D \geq 1$,
    taking 
    \begin{equation}
        \label{eqn:n_condition}
        n > \frac{1}{\ln 2} 
            \ln
            \left(
                \displaystyle
                \frac{\mathcal{R}_0 ^ D - 1}{
                    \beta_V
                    N_V ^ {\infty} +
                    \beta_H N_H
                }
            \right),
    \end{equation}
    we obtain the positivity of $x^{\dagger}$.
\end{proof}
%
%
%
Now we establish our threshold extinction result.
\begin{theorem}
    If \Cref{ass:extinction} hold, then the solution of model 
    \eqref{eqn:sto_vector_host} satisfies
    \begin{equation}
        \limsup_{t \to \infty}
            \frac{1}{t}
            \ln(aI_V + b I_H) < c , \qquad a.s.,
    \end{equation}
    that is, the disease will extinguishes wit probability one.
\end{theorem}
%
%
\begin{proof}
    The main idea is to apply \Cref{prp:quadratic_bound} and 
    \Cref{ass:extinction} to the It\^{o} formula of $\rho(S_V, I_V, S_H, 
    I_H)           $.
        The It\^{o} formula gives
    \begin{equation} \label{eqn:ito_extinction} 
        \begin{aligned}
          d \ln \rho =&
            \left \{
                \dfrac{
                    a ( S_V I_H - \mu_V I_V)
                    +
                    b [\beta_H S_H I_V - \mu_H I_H]
                }{a I_V + bI_H}
            \right \} dt
            \\
            & 
            -
            \frac{\mu_V  a I_V + \mu_H b I_H}{a I_V + b I_H}
            dt
            -
            \frac{1}{2}
            \left \{
                \dfrac{
                    \left(
                        a \sigma_V S_V I_H
                    \right) ^ 2
                    +
                    \left[
                        b \sigma_H S_H I_V
                    \right] ^2
                }{
                    \left( 
                        a I_V + bI_H
                    \right) ^ 2
                }
            \right \} dt
            \\
            & +
            \sigma_V 
            \frac{S_V I_H}{a I_V + b I_H}
            d B_t ^ V
            +
            \sigma_H 
            \frac{S_H I_V}{a I_V + b I_H}
            d B_t ^ H .
        \end{aligned}
    \end{equation}
%%%%%%%%%%%%%%%%%%%%%%%%%%%%%%%%%%%%%%%%%%%%%%%%%%%%%%%%%%%%%%%%%%%%%%%%%%%%%%%%
    \Cref{prp:quadratic_bound} and \textsc{\textbf{(E\textendash 1)}} of 
    \Cref{ass:extinction} imply that
    \begin{equation} 
        \label{eqn:ito_extinction_bound} 
        \begin{aligned}
          d \ln \rho 
            \leq &
            \dfrac{
                \max\{
                    a, b
                \}
            }{
                \min\{
                    a, b
                \}
            }
            \left (
                 \beta_{V} N_V ^ \infty
                +
                \beta_H N_H
            \right ) dt
            \\
            & 
            -
            \frac{1}{2}
            \dfrac{
                \max\{
                    a, b
                \}
            }{
                \min\{
                    a, b
                \}
            }
            \left(
                \sigma_V ^ 2
                +
                \sigma_H ^ 2
            \right) dt
            \\
            &
            -
            \frac{
                \mu_V \mu_H (I_V + I_H)
            }{\max \{a, b\} (I_V + I_H)}
            dt
            \\
            & +
            \sigma_V 
            \frac{S_V I_H}{a I_V + b I_H}
            d B_t ^ V
            +
            \sigma_H
            \frac{S_H I_V}{a I_V + b I_H}
            d B_t ^ H
            %
            \\
            %
            % d \ln \rho(t)
            \leq &
                \underbrace{
                    \left[
                        \dfrac{
                            \max\{a, b\}
                        }{
                            \min\{a, b\}
                        }
                        \left (
                             \beta_V N_V ^ \infty
                            +
                            \beta_H N_H
                        \right )
                        -
                        \frac{
                            \mu_V \mu_H 
                        }{
                            \max \{a, b\} 
                        }
                     \right]
                }_{:= T_1}
                 dt
            \\
            &
            -
            \frac{1}{2}
                \dfrac{
                    \min\{
                        a, b
                    \}
                }{
                    \min\{
                        a, b
                    \}
                }
            \left (
                \sigma_V ^ 2
                +
                \sigma_H ^ 2
            \right ) dt
            \\
            & +
            \sigma_V 
            \frac{S_V I_H}{a I_V + b I_H}
            d B_t ^ V
            +
            \sigma_H
            \frac{S_H I_V}{a I_V + b I_H}
            d B_t ^ H.
        \end{aligned}
    \end{equation}
    According to inequality \eqref{eqn:ito_extinction_bound}, we establish
    a convenient equation for term $T_1$. 
    Let $x:= \max \{ a, b\}$ and $y := \min \{a,b\}$. Then we set
    \begin{equation}
        \label{eqn:quadratic_equation_T_1}
        \left (
            \beta_V
             N_V ^ \infty
            +
            \beta_H N_H
        \right )\frac{x}{y}
        -
        \frac{\mu_V \mu_H}{x} 
        = \mathcal{R}_0^D - 1,
    \end{equation}
    taking $y := \dfrac{x}{2^n}$, with $n \in \mathbb{N}$ 
    %
    % satisfying \eqref{eqn:n_condition}, 
     we rewrite equation
    \eqref{eqn:quadratic_equation_T_1} as
    \begin{equation}
        \label{eqn:quadratic_equation_T_1_quadratic}
        2 ^ n
        \left (
            \beta_V
             N_V ^ \infty
            +
            \beta_H N_H
        \right )
        -
        \frac{\mu_V \mu_H}{x}
        = \mathcal{R}_0^D - 1 . 
    \end{equation}
    By \Cref{lem:quadratic_equation}, 
    taking $n$ 
    $x^\dagger$ is a positive solution
    of \eqref{eqn:quadratic_equation_T_1_quadratic}
     \todo{finish the argument}
    $$
        a:= x^\dagger, \quad b:= \frac{1}{2^n} x^{\dagger},
    $$
     we get 
    \begin{equation}\label{eqn:T_1_rzero_d}
         \dfrac{
             \max\{a, b\}
         }{
             \min\{a, b\}
         }
         \left (
              N_V ^ \infty
             +
             \beta_H N_H
         \right )
         -
         \frac{
             \mu_V \mu_H 
         }{
             \max \{a, b\} 
         }
         = \mathcal{R}_0 ^ D - 1 .
    \end{equation}
    Hence, combining \eqref{eqn:ito_extinction_bound} and 
    \eqref{eqn:T_1_rzero_d},
    applying the mean operator
    $$
        \frac{1}{t} 
            \int_{0}^{t}
                \cdot \ 
            du
    $$
     we obtain
%
    \begin{equation} 
    \label{eqn:rzero_d_bound}
        \begin{aligned}
            \frac{
                \ln \rho(t)
            }{t}
            \leq &
                \underbrace{
                    \mathcal{R}_0 ^ D
                    -
                    \frac{1}{2}
                    \left [
                        \left(
                            \sigma_V N_V ^ \infty
                        \right) ^ 2
                        +
                        \left(
                            \sigma_H N_H
                        \right) ^2
                    \right ] - 1
                }_{=\mathcal{R}_0 ^ S - 1 < 0}
                \\
                & +
                \underbrace{
                    \frac{1}{t}
                    \int_{0}^{t}
                        \sigma_V 
                        \frac{S_V I_H}{a I_V + b I_H}
                    d B_u ^ V
               }_{:=M_t^V}
                +
               \underbrace{
                    \frac{1}{t}
                    \int_{0}^{t}
                        \sigma_H
                        \frac{S_H I_V}{a I_V + b I_H}
                    d B_u ^ H.
               }_{:= M_t ^ H}
        \end{aligned}
    \end{equation}
    Since $M_t^V$ and $M_t^V$ are martingales, the Strong law of large 
    numbers for martingales {\citet[p. 12, Thm. 3.4]{Mao2007}} shows that
    $$
        \lim_{t \to \infty}
            M_t^V = 0 \text{ and }
        %
        \lim_{t \to \infty}
            M_t^H = 0 
        \quad a.s.
    $$    
    Therefore
    $$
        \limsup_{t \to \infty}
            \frac{\ln \rho}{t} <
            \mathcal{R}_0 ^ S - 1 < 0, \quad a.s.,
    $$
    that is, the disease will extinguishes with probability one.
\end{proof}
